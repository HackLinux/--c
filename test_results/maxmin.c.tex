\subsection{Test 15 - maxmin.c}
\subsubsection{Input File}
\begin{lstlisting}[showstringspaces=false,breaklines=true,backgroundcolor=\color{light-gray}, captionpos=b]
/*Result: 5*/

/*
*	Sample --C max/min integer Programme
*	Henry Thacker
*/

int max(int a, int b) {
	if (a >= b) {
		return a;
	}
	return b;
}

int min(int a, int b) {
	if (a <= b) {
		return a;
	}
	return b;
}


int main() {
	int a = 5, b = 9;
	return min(a, b);
}
\end{lstlisting}\subsubsection{Purpose of Test}

\subsubsection{Expected Result}
The expected output for this test $=$ Result: 5
\subsubsection{Interpreter Result}
The result from the interpreter is: 5
\subsubsection{Generated TAC}
\begin{lstlisting}[showstringspaces=false,breaklines=true,backgroundcolor=\color{light-gray}, captionpos=b]
BeginFn max
_max:
InitFrame 3
PopParam a
PopParam b
FnBody
_t1 = a >= b
If _t1 Goto __if1true
Goto __if1end
__if1true:
Return a
__if1end:
Return b
EndFn
BeginFn min
_min:
InitFrame 3
PopParam a
PopParam b
FnBody
_t2 = a <= b
If _t2 Goto __if2true
Goto __if2end
__if2true:
Return a
__if2end:
Return b
EndFn
BeginFn main
_main:
InitFrame 3
FnBody
a = 5
b = 9
PrepareToCall 2
PushParam b
PushParam a
_t3 = CallFn _min
Return _t3
EndFn

\end{lstlisting}\subsubsection{Generated MIPS Assembly}
\begin{lstlisting}[showstringspaces=false,breaklines=true,backgroundcolor=\color{light-gray}, captionpos=b]
         
# Sun Jan 10 14:14:56 2010

.data
	EOL:	.asciiz "\n"
.text

_max:
	move $s7, $ra	# Store Return address in $s7
	li $a0, 28	# Store the frame size required for this AR
	jal mk_ar
	move $s0, $v0	# Store heap start address in $s0
	lw $a0, ($sp)	# Pop the parameter
	sw $a0, -4($fp)	# Write param into heap
	add $sp, $sp, 4	# Move stack pointer
	lw $a0, ($sp)	# Pop the parameter
	sw $a0, -8($fp)	# Write param into heap
	sw $s7, ($sp)	# Save return address in stack
	lw $t1, -4($fp)	# Load local variable
	lw $t2, -8($fp)	# Load local variable
	sge $t0, $t1, $t2	# $c = $a >= $b
	bne $t0, $zero, __if1true
	j __if1end
__if1true:
	move $v0, $t1	# Assign values
	sw $t0, -12($fp)	# Write out used local variable
	lw $ra, ($sp)	# Get return address
	add $sp, $sp, 4	# Pop return address from stack
	lw $fp, 4($s0)	# Load previous frame ptr
	lw $s0, 8($s0)	# Load dynamic link
	jr $ra	# Jump to $ra
__if1end:
	lw $t0, -8($fp)	# Load local variable
	move $v0, $t0	# Assign values
	lw $ra, ($sp)	# Get return address
	add $sp, $sp, 4	# Pop return address from stack
	lw $fp, 4($s0)	# Load previous frame ptr
	lw $s0, 8($s0)	# Load dynamic link
	jr $ra	# Jump to $ra
_min:
	move $s7, $ra	# Store Return address in $s7
	li $a0, 28	# Store the frame size required for this AR
	jal mk_ar
	move $s0, $v0	# Store heap start address in $s0
	lw $a0, ($sp)	# Pop the parameter
	sw $a0, -4($fp)	# Write param into heap
	add $sp, $sp, 4	# Move stack pointer
	lw $a0, ($sp)	# Pop the parameter
	sw $a0, -8($fp)	# Write param into heap
	sw $s7, ($sp)	# Save return address in stack
	lw $t1, -4($fp)	# Load local variable
	lw $t2, -8($fp)	# Load local variable
	sle $t0, $t1, $t2	# $c = $a <= $b
	bne $t0, $zero, __if2true
	j __if2end
__if2true:
	move $v0, $t1	# Assign values
	sw $t0, -12($fp)	# Write out used local variable
	lw $ra, ($sp)	# Get return address
	add $sp, $sp, 4	# Pop return address from stack
	lw $fp, 4($s0)	# Load previous frame ptr
	lw $s0, 8($s0)	# Load dynamic link
	jr $ra	# Jump to $ra
__if2end:
	lw $t0, -8($fp)	# Load local variable
	move $v0, $t0	# Assign values
	lw $ra, ($sp)	# Get return address
	add $sp, $sp, 4	# Pop return address from stack
	lw $fp, 4($s0)	# Load previous frame ptr
	lw $s0, 8($s0)	# Load dynamic link
	jr $ra	# Jump to $ra
_main:
	move $s7, $ra	# Store Return address in $s7
	li $a0, 28	# Store the frame size required for this AR
	jal mk_ar
	move $s0, $v0	# Store heap start address in $s0
	sub $sp, $sp, 4
	sw $s7, ($sp)	# Save return address in stack
	lw $t0, -4($fp)	# Load local variable
	li $t0, 5
	lw $t1, -8($fp)	# Load local variable
	li $t1, 9
	sub $sp, $sp, 4	# Move stack pointer
	sw $t1, ($sp)	# Write param into stack
	sub $sp, $sp, 4	# Move stack pointer
	sw $t0, ($sp)	# Write param into stack
	sw $t0, -4($fp)	# Write out used local variable
	sw $t1, -8($fp)	# Write out used local variable
	lw $v0, ($s0)	# Point callee to same static link as mine (caller)
	move $a1, $s0	# Pass dynamic link
	jal _min
	move $t0, $v0
	sw $t0, -12($fp)	# Write out used local variable
	lw $ra, ($sp)	# Get return address
	add $sp, $sp, 4	# Pop return address from stack
	lw $fp, 4($s0)	# Load previous frame ptr
	lw $s0, 8($s0)	# Load dynamic link
	jr $ra	# Jump to $ra
# Make a new activation record
# Precondition: $a0 contains total required heap size, $a1 contains dynamic link, $v0 contains static link
# Returns: start of heap address in $v0, heap contains reference to static link and old $fp value
mk_ar:
	move $s1, $v0	# Backup static link in $s1
	li $v0, 9	# Allocate space systemcode
	syscall	# Allocate space on heap
	move $s2, $fp	# Backup old $fp in $s2
	add $fp, $v0, $a0	# $fp = heap start address + heap size
	sw $s1, ($v0)	# Save static link
	sw $s2, 4($v0)	# Save old $fp
	sw $a1, 8($v0)	# Save dynamic link
	sw $a0, 12($v0)	# Save framesize
	jr $ra
	.globl main
main:
	move $a1, $zero	# Zero dynamic link
	move $v0, $zero	# Zero static link
	jal _main
	move $a0, $v0	# Retrieve the return value of the main function
	li $v0, 1	# Print integer
	syscall
	li $v0, 4	# Print string
	la $a0, EOL	# Printing EOL character
	syscall
	li $v0, 10	# System exit
	syscall

\end{lstlisting}\subsubsection{SPIM Output}
\begin{verbatim}
SPIM Version 7.4 of January 1, 2009
Copyright 1990-2004 by James R. Larus (larus@cs.wisc.edu).
All Rights Reserved.
See the file README for a full copyright notice.
Loaded: /opt/local/share/spim/exceptions.s
5
\end{verbatim}
\subsubsection{Test Analysis}
\begin{description}
	\item[Interpreter] The interpreter gave the expected result
	\item[TAC] One of the functions \verb!max! is never called. It should be removed from the generated TAC as an optimisation.
	\item[MIPS] The code produces the correct answer.
\end{description}