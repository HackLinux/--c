\subsection{Test 11 - keytest.c}
\subsubsection{Input File}
\begin{lstlisting}[showstringspaces=false,breaklines=true,backgroundcolor=\color{light-gray}, captionpos=b]
/*Result: 83*/

/*
*	Sample --C Test scope traversal
*	Henry Thacker
*/

int main() {
	int a = 5;
	int b = 9;
	int my_func() {
		int c = 20;		
		int test() {
			int d = 11;
			int test2() {  
				d = 18;
				return c + d + b + a + 13;
			}
			return test2() + d;
		}
		return test();
	}
	return my_func();
}
\end{lstlisting}\subsubsection{Purpose of Test}
This test is devised to test scope traversal within multiple nested functions.

\subsubsection{Expected Result}
The expected output for this test $=$ Result: 83
\subsubsection{Interpreter Result}
The result from the interpreter is: 83
\subsubsection{Generated TAC}
\begin{lstlisting}[showstringspaces=false,breaklines=true,backgroundcolor=\color{light-gray}, captionpos=b]
BeginFn main
_main:
InitFrame 4
FnBody
a = 5
b = 9
BeginFn my_func
_my_func:
InitFrame 3
FnBody
c = 20
BeginFn test
_test:
InitFrame 4
FnBody
d = 11
BeginFn test2
_test2:
InitFrame 4
FnBody
d = 18
_t4 = c + d
_t3 = _t4 + b
_t2 = _t3 + a
_t1 = _t2 + 13
Return _t1
EndFn
PrepareToCall 0
_t6 = CallFn _test2
_t5 = _t6 + d
Return _t5
EndFn
PrepareToCall 0
_t7 = CallFn _test
Return _t7
EndFn
PrepareToCall 0
_t8 = CallFn _my_func
Return _t8
EndFn

\end{lstlisting}\subsubsection{Generated MIPS Assembly}
\begin{lstlisting}[showstringspaces=false,breaklines=true,backgroundcolor=\color{light-gray}, captionpos=b]
         
# Sun Jan 10 14:14:56 2010

.data
	EOL:	.asciiz "\n"
.text

_main:
	move $s7, $ra	# Store Return address in $s7
	li $a0, 32	# Store the frame size required for this AR
	jal mk_ar
	move $s0, $v0	# Store heap start address in $s0
	sub $sp, $sp, 4
	sw $s7, ($sp)	# Save return address in stack
	lw $t0, -4($fp)	# Load local variable
	li $t0, 5
	lw $t1, -8($fp)	# Load local variable
	li $t1, 9
	sw $t0, -4($fp)	# Write out used local variable
	sw $t1, -8($fp)	# Write out used local variable
	move $v0, $s0	# Set this current activation record as the static link
	move $a1, $s0	# Pass dynamic link
	jal _my_func
	move $t0, $v0
	sw $t0, -16($fp)	# Write out used local variable
	lw $ra, ($sp)	# Get return address
	add $sp, $sp, 4	# Pop return address from stack
	lw $fp, 4($s0)	# Load previous frame ptr
	lw $s0, 8($s0)	# Load dynamic link
	jr $ra	# Jump to $ra
_my_func:
	move $s7, $ra	# Store Return address in $s7
	li $a0, 28	# Store the frame size required for this AR
	jal mk_ar
	move $s0, $v0	# Store heap start address in $s0
	sub $sp, $sp, 4
	sw $s7, ($sp)	# Save return address in stack
	lw $t0, -4($fp)	# Load local variable
	li $t0, 20
	sw $t0, -4($fp)	# Write out used local variable
	move $v0, $s0	# Set this current activation record as the static link
	move $a1, $s0	# Pass dynamic link
	jal _test
	move $t0, $v0
	sw $t0, -12($fp)	# Write out used local variable
	lw $ra, ($sp)	# Get return address
	add $sp, $sp, 4	# Pop return address from stack
	lw $fp, 4($s0)	# Load previous frame ptr
	lw $s0, 8($s0)	# Load dynamic link
	jr $ra	# Jump to $ra
_test:
	move $s7, $ra	# Store Return address in $s7
	li $a0, 32	# Store the frame size required for this AR
	jal mk_ar
	move $s0, $v0	# Store heap start address in $s0
	sub $sp, $sp, 4
	sw $s7, ($sp)	# Save return address in stack
	lw $t0, -4($fp)	# Load local variable
	li $t0, 11
	sw $t0, -4($fp)	# Write out used local variable
	move $v0, $s0	# Set this current activation record as the static link
	move $a1, $s0	# Pass dynamic link
	jal _test2
	move $t0, $v0
	lw $t2, -4($fp)	# Load local variable
	add $t1, $t0, $t2
	move $v0, $t1	# Assign values
	sw $t0, -16($fp)	# Write out used local variable
	sw $t1, -12($fp)	# Write out used local variable
	lw $ra, ($sp)	# Get return address
	add $sp, $sp, 4	# Pop return address from stack
	lw $fp, 4($s0)	# Load previous frame ptr
	lw $s0, 8($s0)	# Load dynamic link
	jr $ra	# Jump to $ra
_test2:
	move $s7, $ra	# Store Return address in $s7
	li $a0, 32	# Store the frame size required for this AR
	jal mk_ar
	move $s0, $v0	# Store heap start address in $s0
	sub $sp, $sp, 4
	sw $s7, ($sp)	# Save return address in stack
	lw $t0, ($s0)	# Move up a static link
	lw $t1, 12($t0)	# Load framesize for static link
	add $t1, $t1, $t0	# Seek to $fp [end of AR]
	lw $t0, -4($t1)	# d
	li $t0, 18
	lw $t2, ($s0)	# Move up a static link
	lw $t2, ($t2)	# Move up a static link
	lw $t3, 12($t2)	# Load framesize for static link
	add $t3, $t3, $t2	# Seek to $fp [end of AR]
	lw $t2, -4($t3)	# c
	add $t1, $t2, $t0
	lw $t4, ($s0)	# Move up a static link
	lw $t4, ($t4)	# Move up a static link
	lw $t4, ($t4)	# Move up a static link
	lw $t5, 12($t4)	# Load framesize for static link
	add $t5, $t5, $t4	# Seek to $fp [end of AR]
	lw $t4, -8($t5)	# b
	add $t3, $t1, $t4
	lw $t6, ($s0)	# Move up a static link
	lw $t6, ($t6)	# Move up a static link
	lw $t6, ($t6)	# Move up a static link
	lw $t7, 12($t6)	# Load framesize for static link
	add $t7, $t7, $t6	# Seek to $fp [end of AR]
	lw $t6, -4($t7)	# a
	add $t5, $t3, $t6
	addi $t7, $t5, 13
	move $v0, $t7	# Assign values
	lw $t8, ($s0)	# Move up a static link
	lw $s6, 12($t8)	# Load framesize for static link
	add $s6, $s6, $t8	# Seek to $fp [end of AR]
	sw $t0, -4($s6)	# Save distant modified variable
	sw $t1, -16($fp)	# Write out used local variable
	sw $t3, -12($fp)	# Write out used local variable
	sw $t5, -8($fp)	# Write out used local variable
	sw $t7, -4($fp)	# Write out used local variable
	lw $ra, ($sp)	# Get return address
	add $sp, $sp, 4	# Pop return address from stack
	lw $fp, 4($s0)	# Load previous frame ptr
	lw $s0, 8($s0)	# Load dynamic link
	jr $ra	# Jump to $ra
# Make a new activation record
# Precondition: $a0 contains total required heap size, $a1 contains dynamic link, $v0 contains static link
# Returns: start of heap address in $v0, heap contains reference to static link and old $fp value
mk_ar:
	move $s1, $v0	# Backup static link in $s1
	li $v0, 9	# Allocate space systemcode
	syscall	# Allocate space on heap
	move $s2, $fp	# Backup old $fp in $s2
	add $fp, $v0, $a0	# $fp = heap start address + heap size
	sw $s1, ($v0)	# Save static link
	sw $s2, 4($v0)	# Save old $fp
	sw $a1, 8($v0)	# Save dynamic link
	sw $a0, 12($v0)	# Save framesize
	jr $ra
	.globl main
main:
	move $a1, $zero	# Zero dynamic link
	move $v0, $zero	# Zero static link
	jal _main
	move $a0, $v0	# Retrieve the return value of the main function
	li $v0, 1	# Print integer
	syscall
	li $v0, 4	# Print string
	la $a0, EOL	# Printing EOL character
	syscall
	li $v0, 10	# System exit
	syscall

\end{lstlisting}\subsubsection{SPIM Output}
\begin{verbatim}
SPIM Version 7.4 of January 1, 2009
Copyright 1990-2004 by James R. Larus (larus@cs.wisc.edu).
All Rights Reserved.
See the file README for a full copyright notice.
Loaded: /opt/local/share/spim/exceptions.s
83
\end{verbatim}
\subsubsection{Test Analysis}
\begin{description}
	\item[Interpreter] The interpreter gave the expected result
	\item[MIPS] The code produces the correct answer, but there is a lot of function calling overhead and static link traversal.
\end{description}

