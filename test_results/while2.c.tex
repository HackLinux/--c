\subsection{Test 22 - while2.c}
\subsubsection{Input File}
\begin{lstlisting}[showstringspaces=false,breaklines=true,backgroundcolor=\color{light-gray}, captionpos=b]
/*Result: 3*/

/*
*	Sample --C While Loop #2 Programme
*	Henry Thacker
*/

/* Main entry point */
int main() {
	int i = 5;
	while (i > 1)  {
		i = i - 1;
		if (i==4) {
			continue;
		}
		return i;
	}
	return 5;
}
\end{lstlisting}\subsubsection{Purpose of Test}

\subsubsection{Expected Result}
The expected output for this test $=$ Result: 3
\subsubsection{Interpreter Result}
The result from the interpreter is: 3
\subsubsection{Generated TAC}
\begin{lstlisting}[showstringspaces=false,breaklines=true,backgroundcolor=\color{light-gray}, captionpos=b]
BeginFn main
_main:
InitFrame 4
FnBody
i = 5
__while1:
_t1 = i > 1
If _t1 Goto __if1true
Goto __while1end
__if1true:
_t2 = i - 1
i = _t2
_t3 = i == 4
If _t3 Goto __if2true
Goto __if2end
__if2true:
Goto __while1
__if2end:
Return i
Goto __while1
__if1end:
__while1end:
Return 5
EndFn

\end{lstlisting}\subsubsection{Generated MIPS Assembly}
\begin{lstlisting}[showstringspaces=false,breaklines=true,backgroundcolor=\color{light-gray}, captionpos=b]
         
# Sun Jan 10 14:14:57 2010

.data
	EOL:	.asciiz "\n"
.text

_main:
	move $s7, $ra	# Store Return address in $s7
	li $a0, 32	# Store the frame size required for this AR
	jal mk_ar
	move $s0, $v0	# Store heap start address in $s0
	sub $sp, $sp, 4
	sw $s7, ($sp)	# Save return address in stack
	lw $t0, -4($fp)	# Load local variable
	li $t0, 5
__while1:
	li $t2, 1
	sgt $t1, $t0, $t2	# $c = $a > $b
	bne $t1, $zero, __if1true
	j __while1end
__if1true:
	sub $t3, $t0, 1
	move $t0, $t3	# Assign values
	li $t5, 4
	seq $t4, $t0, $t5	# $c = $a == $b
	bne $t4, $zero, __if2true
	j __if2end
__if2true:
	j __while1
__if2end:
	move $v0, $t0	# Assign values
	sw $t0, -4($fp)	# Write out used local variable
	sw $t1, -8($fp)	# Write out used local variable
	sw $t3, -12($fp)	# Write out used local variable
	sw $t4, -16($fp)	# Write out used local variable
	lw $ra, ($sp)	# Get return address
	add $sp, $sp, 4	# Pop return address from stack
	lw $fp, 4($s0)	# Load previous frame ptr
	lw $s0, 8($s0)	# Load dynamic link
	jr $ra	# Jump to $ra
	j __while1
__if1end:
__while1end:
	li $v0, 5
	lw $ra, ($sp)	# Get return address
	add $sp, $sp, 4	# Pop return address from stack
	lw $fp, 4($s0)	# Load previous frame ptr
	lw $s0, 8($s0)	# Load dynamic link
	jr $ra	# Jump to $ra
# Make a new activation record
# Precondition: $a0 contains total required heap size, $a1 contains dynamic link, $v0 contains static link
# Returns: start of heap address in $v0, heap contains reference to static link and old $fp value
mk_ar:
	move $s1, $v0	# Backup static link in $s1
	li $v0, 9	# Allocate space systemcode
	syscall	# Allocate space on heap
	move $s2, $fp	# Backup old $fp in $s2
	add $fp, $v0, $a0	# $fp = heap start address + heap size
	sw $s1, ($v0)	# Save static link
	sw $s2, 4($v0)	# Save old $fp
	sw $a1, 8($v0)	# Save dynamic link
	sw $a0, 12($v0)	# Save framesize
	jr $ra
	.globl main
main:
	move $a1, $zero	# Zero dynamic link
	move $v0, $zero	# Zero static link
	jal _main
	move $a0, $v0	# Retrieve the return value of the main function
	li $v0, 1	# Print integer
	syscall
	li $v0, 4	# Print string
	la $a0, EOL	# Printing EOL character
	syscall
	li $v0, 10	# System exit
	syscall

\end{lstlisting}\subsubsection{SPIM Output}
\begin{verbatim}
SPIM Version 7.4 of January 1, 2009
Copyright 1990-2004 by James R. Larus (larus@cs.wisc.edu).
All Rights Reserved.
See the file README for a full copyright notice.
Loaded: /opt/local/share/spim/exceptions.s
3
\end{verbatim}\subsubsection{Test Analysis}

