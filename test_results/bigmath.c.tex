\subsection{Test 1 - bigmath.c}
\subsubsection{Input File}
\begin{lstlisting}[showstringspaces=false,breaklines=true,backgroundcolor=\color{light-gray}, captionpos=b]
/*Result: 2034*/


/*
*	Sample --C Test large number of variable assignments
*	Henry Thacker
*/

int main() {
	int a = 5, b = 10;
	int c = 929, d = 32;
	int g;
	int e = 90;
	int f = 72;
	int h = 442;
	g = 12;
	return a + b + c + d + e + f + g + (2 * h); 
}
\end{lstlisting}\subsubsection{Purpose of Test}
This test was created to see what happens when registers need to be spilled into the activation record.

\subsubsection{Expected Result}
The expected output for this test $=$ Result: 2034
\subsubsection{Interpreter Result}
The result from the interpreter is: 2034
\subsubsection{Generated TAC}
\begin{lstlisting}[showstringspaces=false,breaklines=true,backgroundcolor=\color{light-gray}, captionpos=b]
BeginFn main
_main:
InitFrame 16
FnBody
a = 5
b = 10
c = 929
d = 32
e = 90
f = 72
h = 442
g = 12
_t7 = a + b
_t6 = _t7 + c
_t5 = _t6 + d
_t4 = _t5 + e
_t3 = _t4 + f
_t2 = _t3 + g
_t8 = 2 * h
_t1 = _t2 + _t8
Return _t1
EndFn

\end{lstlisting}\subsubsection{Generated MIPS Assembly}
\begin{lstlisting}[showstringspaces=false,breaklines=true,backgroundcolor=\color{light-gray}, captionpos=b]
         
# Sun Jan 10 14:14:55 2010

.data
	EOL:	.asciiz "\n"
.text

_main:
	move $s7, $ra	# Store Return address in $s7
	li $a0, 80	# Store the frame size required for this AR
	jal mk_ar
	move $s0, $v0	# Store heap start address in $s0
	sub $sp, $sp, 4
	sw $s7, ($sp)	# Save return address in stack
	lw $t0, -4($fp)	# Load local variable
	li $t0, 5
	lw $t1, -8($fp)	# Load local variable
	li $t1, 10
	lw $t2, -12($fp)	# Load local variable
	li $t2, 929
	lw $t3, -16($fp)	# Load local variable
	li $t3, 32
	lw $t4, -24($fp)	# Load local variable
	li $t4, 90
	lw $t5, -28($fp)	# Load local variable
	li $t5, 72
	lw $t6, -32($fp)	# Load local variable
	li $t6, 442
	lw $t7, -20($fp)	# Load local variable
	li $t7, 12
	add $t8, $t0, $t1
	add $t9, $t8, $t2
	sw $t3, -16($fp)	# Write out used local variable
	sw $t4, -24($fp)	# Write out used local variable
	lw $t4, -16($fp)	# Load local variable
	add $t3, $t9, $t4
	sw $t5, -28($fp)	# Write out used local variable
	sw $t6, -32($fp)	# Write out used local variable
	lw $t6, -24($fp)	# Load local variable
	add $t5, $t3, $t6
	sw $t7, -20($fp)	# Write out used local variable
	sw $t0, -4($fp)	# Write out used local variable
	lw $t0, -28($fp)	# Load local variable
	add $t7, $t5, $t0
	sw $t1, -8($fp)	# Write out used local variable
	sw $t8, -60($fp)	# Write out used local variable
	lw $t8, -20($fp)	# Load local variable
	add $t1, $t7, $t8
	sw $t2, -12($fp)	# Write out used local variable
	sw $t9, -56($fp)	# Write out used local variable
	li $t9, 2
	lw $t4, -32($fp)	# Load local variable
	mult $t9, $t4
	mflo $t2
	sw $t3, -52($fp)	# Write out used local variable
	add $t3, $t1, $t2
	move $v0, $t3	# Assign values
	sw $t1, -40($fp)	# Write out used local variable
	sw $t2, -64($fp)	# Write out used local variable
	sw $t3, -36($fp)	# Write out used local variable
	sw $t5, -48($fp)	# Write out used local variable
	sw $t7, -44($fp)	# Write out used local variable
	lw $ra, ($sp)	# Get return address
	add $sp, $sp, 4	# Pop return address from stack
	lw $fp, 4($s0)	# Load previous frame ptr
	lw $s0, 8($s0)	# Load dynamic link
	jr $ra	# Jump to $ra
# Make a new activation record
# Precondition: $a0 contains total required heap size, $a1 contains dynamic link, $v0 contains static link
# Returns: start of heap address in $v0, heap contains reference to static link and old $fp value
mk_ar:
	move $s1, $v0	# Backup static link in $s1
	li $v0, 9	# Allocate space systemcode
	syscall	# Allocate space on heap
	move $s2, $fp	# Backup old $fp in $s2
	add $fp, $v0, $a0	# $fp = heap start address + heap size
	sw $s1, ($v0)	# Save static link
	sw $s2, 4($v0)	# Save old $fp
	sw $a1, 8($v0)	# Save dynamic link
	sw $a0, 12($v0)	# Save framesize
	jr $ra
	.globl main
main:
	move $a1, $zero	# Zero dynamic link
	move $v0, $zero	# Zero static link
	jal _main
	move $a0, $v0	# Retrieve the return value of the main function
	li $v0, 1	# Print integer
	syscall
	li $v0, 4	# Print string
	la $a0, EOL	# Printing EOL character
	syscall
	li $v0, 10	# System exit
	syscall

\end{lstlisting}\subsubsection{SPIM Output}
\begin{verbatim}
SPIM Version 7.4 of January 1, 2009
Copyright 1990-2004 by James R. Larus (larus@cs.wisc.edu).
All Rights Reserved.
See the file README for a full copyright notice.
Loaded: /opt/local/share/spim/exceptions.s
2034
\end{verbatim}
\subsubsection{Test Analysis}
\begin{description}
	\item[Interpreter] The interpreter gave the expected result
	\item[MIPS] The MIPS code writes variables out to memory quite a bit. The MIPS code just follows the simple rule where any spilled register should be written out to memory if it contains a modified value. Without any kind of liveness checking, it is difficult for the compiler to take more intelligent decisions. The code produces the correct answer.
\end{description}