\subsection{Test 8 - iftest.c}
\subsubsection{Input File}
\begin{lstlisting}[showstringspaces=false,breaklines=true,backgroundcolor=\color{light-gray}, captionpos=b]
/*Result: 9*/

/*
*	Sample --C Nested IF Programme
*	Henry Thacker
*/

int main() {
	int a = 15;
	if (1) {
		if (0) {
			a = 5;
		}
		else {
			a = 12;
		}
	}
	else {
		a = 9;
	}
	if (1) {
		if (0) {
			a = a - 1;
		}
		else {
			a = a - 3;
		}
	}
	else {
		a = a - 3;
	}
	return a;
}
\end{lstlisting}\subsubsection{Purpose of Test}
This code is designed to test that branching (including nested IF / ELSE statements) operations work as intended.

\subsubsection{Expected Result}
The expected output for this test $=$ Result: 9
\subsubsection{Interpreter Result}
The result from the interpreter is: 9
\subsubsection{Generated TAC}
\begin{lstlisting}[showstringspaces=false,breaklines=true,backgroundcolor=\color{light-gray}, captionpos=b]
BeginFn main
_main:
InitFrame 4
FnBody
a = 15
If 1 Goto __if1true
a = 9
Goto __if1end
__if1true:
If 0 Goto __if2true
a = 12
Goto __if2end
__if2true:
a = 5
__if2end:
__if1end:
If 1 Goto __if3true
_t1 = a - 3
a = _t1
Goto __if3end
__if3true:
If 0 Goto __if4true
_t2 = a - 3
a = _t2
Goto __if4end
__if4true:
_t3 = a - 1
a = _t3
__if4end:
__if3end:
Return a
EndFn

\end{lstlisting}\subsubsection{Generated MIPS Assembly}
\begin{lstlisting}[showstringspaces=false,breaklines=true,backgroundcolor=\color{light-gray}, captionpos=b]
         
# Sun Jan 10 14:14:55 2010

.data
	EOL:	.asciiz "\n"
.text

_main:
	move $s7, $ra	# Store Return address in $s7
	li $a0, 32	# Store the frame size required for this AR
	jal mk_ar
	move $s0, $v0	# Store heap start address in $s0
	sub $sp, $sp, 4
	sw $s7, ($sp)	# Save return address in stack
	lw $t0, -4($fp)	# Load local variable
	li $t0, 15
	li $t1, 1
	bne $t1, $zero, __if1true
	li $t0, 9
	j __if1end
__if1true:
	bne $zero, $zero, __if2true
	li $t0, 12
	j __if2end
__if2true:
	li $t0, 5
__if2end:
__if1end:
	li $t2, 1
	bne $t2, $zero, __if3true
	sub $t3, $t0, 3
	move $t0, $t3	# Assign values
	j __if3end
__if3true:
	bne $zero, $zero, __if4true
	sub $t4, $t0, 3
	move $t0, $t4	# Assign values
	j __if4end
__if4true:
	sub $t5, $t0, 1
	move $t0, $t5	# Assign values
__if4end:
__if3end:
	move $v0, $t0	# Assign values
	sw $t0, -4($fp)	# Write out used local variable
	sw $t3, -8($fp)	# Write out used local variable
	sw $t4, -12($fp)	# Write out used local variable
	sw $t5, -16($fp)	# Write out used local variable
	lw $ra, ($sp)	# Get return address
	add $sp, $sp, 4	# Pop return address from stack
	lw $fp, 4($s0)	# Load previous frame ptr
	lw $s0, 8($s0)	# Load dynamic link
	jr $ra	# Jump to $ra
# Make a new activation record
# Precondition: $a0 contains total required heap size, $a1 contains dynamic link, $v0 contains static link
# Returns: start of heap address in $v0, heap contains reference to static link and old $fp value
mk_ar:
	move $s1, $v0	# Backup static link in $s1
	li $v0, 9	# Allocate space systemcode
	syscall	# Allocate space on heap
	move $s2, $fp	# Backup old $fp in $s2
	add $fp, $v0, $a0	# $fp = heap start address + heap size
	sw $s1, ($v0)	# Save static link
	sw $s2, 4($v0)	# Save old $fp
	sw $a1, 8($v0)	# Save dynamic link
	sw $a0, 12($v0)	# Save framesize
	jr $ra
	.globl main
main:
	move $a1, $zero	# Zero dynamic link
	move $v0, $zero	# Zero static link
	jal _main
	move $a0, $v0	# Retrieve the return value of the main function
	li $v0, 1	# Print integer
	syscall
	li $v0, 4	# Print string
	la $a0, EOL	# Printing EOL character
	syscall
	li $v0, 10	# System exit
	syscall

\end{lstlisting}\subsubsection{SPIM Output}
\begin{verbatim}
SPIM Version 7.4 of January 1, 2009
Copyright 1990-2004 by James R. Larus (larus@cs.wisc.edu).
All Rights Reserved.
See the file README for a full copyright notice.
Loaded: /opt/local/share/spim/exceptions.s
9
\end{verbatim}
\subsubsection{Test Analysis}
\begin{description}
	\item[Interpreter] The interpreter gave the expected result
	\item[TAC] There are cases where a couple of different labels are defined next to each other (i.e. \verb!__if2end! and \verb!__if1end!). The reference to one of these labels could be removed (and all references to it refactored) to save extra lines in the output. This change would benefit the code generation stage too.
	\item[MIPS] The code produces the correct answer.
\end{description}
