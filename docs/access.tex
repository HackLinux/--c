\chapter{Access to source and binaries}

\section{Directory}
/u/s/ht221/cm30171 (correct UNIX permissions have been set to allow read / execute access).

\section{Host details}
Host tested on: limol.bath.ac.uk (part of the LCPU cluster)

\section{Folder Structure}
There are three subfolders in the main source directory, as follows:
\ \\
\begin{description}
	\item[debug] A copy of the source code and binaries, with debug information and more verbose output for both MIPS and during interpretation
	\item[release] A copy of the source code and binaries, with all debugging information stripped and the minimum verbosity output
	\item[tests] A series of tests are available within the examples subdirectory. This ``examples" folder is symlinked as examples from both the debug and release folders as a convenient way to execute test cases. The directory also contains a test harness that I developed, although unfortunately the LCPU machines do not have the correct ruby gems installed to see this in action.
\end{description}

\section{Instructions}
The interpreter, tac-generator and compiler are all encompassed within the \verb!mycc! binary. Invoking \verb!mycc! without any parameters shows the usage switches, which are shown in listing \ref{lst:mycc-switchesex}. \textbf{Please note:} some examples will not work in MIPS due to some shortcomings that I will mention later on. Such examples may induce undefined behaviour / results.
\ \\ \ \\
So, to interpret the \verb1fibonacci.c1 example, one could run:

\begin{lstlisting}[showstringspaces=false,language=java,breaklines=true, backgroundcolor=\color{light-gray}, captionpos=b, label={lst:mycc-fibex}]
	./mycc -i < examples/fibonacci.c
\end{lstlisting}

\ \\Or to generate the three address code representation of \verb1twice.c1:
\begin{lstlisting}[showstringspaces=false,language=java,breaklines=true, backgroundcolor=\color{light-gray}, captionpos=b, label={lst:mycc-twiceex}]
	./mycc -t < examples/twice.c
\end{lstlisting}

\begin{lstlisting}[showstringspaces=false,language=java,breaklines=true,caption={mycc options}, backgroundcolor=\color{light-gray}, captionpos=b, label={lst:mycc-switchesex}]
henry-thackers-macbook-pro:Compilers henry$ ./mycc
--C Compiler - Input syntax

./mycc [-p|-i|-t|-m] < input_source

-p = parse mode - print out the parse tree and terminate
-i = interpret - interpret the parse tree and print the result
-t = TAC generator - print out TAC representation of the programme
-m = MIPS generator - generate MIPS machine code
\end{lstlisting}

\section{Source control}
The source code can be looked at in a far more convenient manner, with syntax highlighting, etc on GitHub at: \url{http://github.com/henrythacker/--c}.

