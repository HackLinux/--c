\chapter{Overview}

The main aim of this project was to build on top of an existing lexer and parser for the \mmc language to provide: an interpreter, a form of intermediate representation (three address code) and finally, a compiler to MIPS assembler. \mmc is a fictitious language that was developed for the purposes of this coursework. The language has several interesting features that had profound effects on the implementation of the project. A few of the pertinent points are mentioned below:

\begin{itemize}
	\item The language allows the definition of inner functions that capture the scope of their definition environment
	\item Function typed variables are permitted and can be returned as function results or passed as a parameter (higher-order functions).
	\item No strings, character types, structures, floating point numbers or pointers. Integer, void and function typed variables are the only ones supported.
\end{itemize}
\vspace{-0.25cm}\ \\
The coursework was completed in several separate stages and developed using C, to simplify the development. At early stages of the project, both C++ and Ruby were seriously considered as implementation languages due to the benefits of using object orientation and the sheer scale of their respective standard libraries. It was decided, however, that the amount of time-overhead involved in interfacing with the existing code would negate much of the benefit and potentially introduce bugs or unknown complexity.
\ \\ \ \\
Each of the different parts of the project are split into separate implementation files and folders, but are built as one large binary. Communication between different sections of the interpreter and compiler is achieved through use of shared header files. This makes it easy to use of the output of one part of the compiler (or interpreter) as the input in the next part of the process.
\ \\ \ \\
The implementation is fairly complete, but for a few exceptional cases within in the MIPS code generator. These are only very minor issues and indeed all of the given coursework examples (and many of my own creation) work flawlessly throughout the entire pipeline. Throughout the implementation, various text books, web resources and lecture notes were referred to (mainly for the code generation stage). Where required, in this document, the findings from each resource will be attributed.

